%\documentclass[14pt,xcolor=dvipsnames]{beamer}
\documentclass{beamer}

%主题
\usetheme{Madrid}
%\usetheme{PaloAlto}
%\usetheme{Berkeley}
%\usetheme{Berlin}
%\usetheme{Boadilla}
%\usetheme{AnnArbor}
%\usetheme{CambridgeUS}
%\usetheme{Darmstadt}
%\useoutertheme{umbcfootline}
%\setbeamertemplate{items}[default]
%\setbeamertemplate{blocks}[rounded][shadow=true]
%\usecolortheme{default}
% \usecolortheme{dolphin}
\setbeamertemplate{enumerate items}[default]
\setbeamertemplate{itemize items}[default]

%xeCJK相关宏包
%字体设置
\usepackage{xltxtra,fontspec,xunicode}
\usepackage[slantfont,boldfont,CJKchecksingle]{xeCJK}
\CJKsetecglue{\hskip 0.15em plus 0.05 em minus 0.05em}
\XeTeXlinebreaklocale "zh"
\XeTeXlinebreakskip=0pt plus 1pt minus 0.1pt
%中文字体
%\setCJKmainfont[BoldFont=Adobe Heiti Std]{Adobe Song Std}
\setCJKmainfont{WenQuanYi Micro Hei}
%\setCJKsansfont[BoldFont=Adobe Heiti Std]{Adobe Kaiti Std}
\setCJKsansfont{WenQuanYi Micro Hei}
%\setCJKmonofont{WenQuanYi Micro Hei Mono}
\setCJKmonofont{WenQuanYi Micro Hei Mono}
%\usefonttheme{professionalfonts}
%\usefonttheme{serif}
%定义新字体
\setCJKfamilyfont{song}{Adobe Song Std}
\setCJKfamilyfont{kai}{Adobe Kaiti Std}
\setCJKfamilyfont{hei}{Adobe Heiti Std}
\setCJKfamilyfont{fangsong}{Adobe Fangsong Std}
\setCJKfamilyfont{lisu}{LiSu}
\setCJKfamilyfont{youyuan}{YouYuan}
\newcommand{\song}{\CJKfamily{song}}
\newcommand{\kai}{\CJKfamily{kai}}
\newcommand{\hei}{\CJKfamily{hei}}
\newcommand{\fangsong}{\CJKfamily{fangsong}}
\newcommand{\lisu}{\CJKfamily{lisu}}
\newcommand{\youyuan}{\CJKfamily{youyuan}}

\defaultfontfeatures{Mapping=tex-text}

\renewcommand\figurename{图}
\renewcommand\tablename{表}
\newcommand{\ud}{\mathrm{d}}

\usepackage{setspace}
\usepackage{colortbl,xcolor,color}
\usepackage{hyperref}
\hypersetup
%{xetex,bookmarksnumbered=true,bookmarksopen,pdfborder=1,breaklinks,colorlinks=true,linkcolor=cyan,filecolor=black,urlcolor=cyan,citecolor=green}
%{xetex,bookmarksnumbered=true,bookmarksopen,pdfborder=1,breaklinks,colorlinks=true,filecolor=black,urlcolor=cyan,citecolor=green}
{xetex,bookmarksnumbered=true,bookmarksopen,pdfborder=1,breaklinks,colorlinks=true,CJKbookmarks=true}
%\setlength{\parindent}{2em}

%插入图片
\usepackage{graphicx}
\graphicspath{{fig/}}

%可能用到的包
\usepackage{amsmath,amssymb}
\usepackage{multimedia}
\usepackage{tabularx,multirow,multicol,keystroke,subfigure,longtable}
\usepackage{diagbox}
\usepackage[all]{xy}
\usepackage[backend=biber,style=caspervector,utf8,sorting=centy]{biblatex}
\addbibresource{ref.bib}

\usepackage[absolute,overlay]{textpos}

\usepackage{tikz}
\usetikzlibrary{shapes.geometric}
\usetikzlibrary{trees}

\usepackage{booktabs}

%\usepackage{cite}
%\usepackage[numbers,sort&compress]{natbib}
%\usepackage{hypernat}
%\bibliographystyle{plain}

%在表格、图片等的标题中显示编号
\setbeamertemplate{caption}[numbered]

%插入目录
\AtBeginSection[]
{
    \begin{frame}
        \frametitle{目录}
        \tableofcontents[currentsection,hideallsubsections]
    \end{frame}
    \addtocounter{framenumber}{-1}
}

\AtBeginSubsection[]
{
    \begin{frame}{子目录}
        %\setcounter{tocdepth}{2}
        \tableofcontents[currentsection,currentsubsection,sections={\thesection}]
    \end{frame}
    \addtocounter{framenumber}{-1}
}
